%!TEX root = ../ausarbeitung.tex
\section*{Anleitung}
\addcontentsline{toc}{section}{Anleitung}
Im folgendem wird eine kurze Anleitung zur Verwendung des Programms gegeben.
\subsection*{Systemvoraussetzungen}
Zur Verwendung des Codes wird eine Python 3.6-Umgebung vorausgesetzt. Die Installation von Python 3.6 wird auf der offiziellen Homepage beschrieben \url{https://www.python.org/}. Des Weiteren wird zur Sequenzierung das ExifTool von Phil Harvey in Version 11.33 verwendet. Die Installation für verschiedene Betriebssysteme wird auf der entsprechenden Homepage beschrieben \url{http://www.sno.phy.queensu.ca/~phil/exiftool/}. Zusätzlich müssen die folgenden Python-Pakete installiert sein: Matplotlib (Version 3.0.2), Numpy (Version 1.15.4), Scipy (1.2.1),  OpenCV mit \texttt{opencv-contrib-python} (Version 3.4.2.17), python-dateutil (Version 2.7.5), Scikit-image (Version 0.14.2), Scikit-learn (Version 0.20.2) und Numba (Version 0.43.0).

\subsection*{Sequenzierung}
\subsection*{Lokalisierung}
Nach der Sequenzierung von Bildern könnten die Sequenzen mit \textit{Hintergrund-Subtraktion} und \textit{Hintergrundapproximation} lokalisiert bzw. segmentiert werden. Dafür kann die Methode \textit{``segment''} in der Klasse \textit{``segment.py''} verwendet werden.
Die Parametern sind der Pfad des Überordners, der alle Unterordner von Sequenzen enthält, die ``label'' von jeweiligen Tier und der Ausgabepfad von den Segmentierten Bildern.\\\\
Für die Lokalisierung mit \textit{Sliding-Windows} und PCA-Klassifikator man kann die Methode \textit{``TrainingsPhase''} in der Klasse \textit{``pca\_knn.py''} anpassen, um die Pfade von den Schnittbildern festzulegen. Die Schnittbilder sollen möglichst quadratische Maßen haben und das Tier enthalten. Defaultmäßig wird das trainiertes \textit{``pca.sav und knn.sav''} Model geladen. Anschließend kann das gesuchte Tier in den Bildern mithilfe von der Methode \textit{``localisation''} gefunden und ausgegeben werden. Analog kann die Lokalisierung mit \textit{Sliding-Windows} und HOG-Klassifikator in der Klasse \textit{``hog\_svm.py''} verwendet werden.
\subsection*{Klassifizierung mit HOG}
\subsection*{SPM}




