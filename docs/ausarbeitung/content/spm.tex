%!TEX root = ../ausarbeitung.tex
\section{Klassifizierung mit Spatial Pyramid Matching}

\subsection{Grundidee Spatial Pyramid Matching}

-hintergrund: Lazebnik et al., Scene Classification
-aufteilung in immer feinere regionen
-bild mit dense features und spatial pyramid einfügen
-features für alle regionen (sift, lbp)
-berechnungen von codes aufgrund dieser features (enkodierung, pooling, normalisierung)
-trainieren einer SVM mit den codes

\subsection{Locality-constrained Linear Coding}

-kurze einleitung
-vortraining des codebooks mit clustering
-vergleich mit vector quantization
-gleichung
-lösungsalgorithmus
-pooling
-normalisierung
-lineare separierarkeit, laufzeit, bezug auf hog kapitel
-kurze erwähnung der svm mit squared-hinge-loss und 

\subsection{Implementierung}

-umsetzung numpy, optimierung numba
-opencv für sift, scikit-image für lbp
-clustering und linear svm mit scikit-learn
-transformer mit scikit-learn, einbettung in pipeline