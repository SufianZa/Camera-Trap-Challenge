%!TEX root = ausarbeitung.tex
% Author: Phil Steinhorst, p.st@wwu.de
\documentclass[%
	paper=a4,
	fontsize=11,
	DIV=13,
	BCOR=9mm,
	chapterprefix=false,
	headings=optiontohead,
	ngerman,
	final=true,
	twoside,
	cleardoublepage=empty
]{scrartcl}

% Basics für Codierung und Sprache
% ===========================================================
	\usepackage{scrtime}
	\usepackage{etex}
	\usepackage{shellesc}
	\usepackage[final]{graphicx}
	\usepackage[utf8]{inputenc}
	\usepackage{babel}
	\usepackage[german=quotes]{csquotes}
	\usepackage{mathtools}
% ===========================================================

% Fonts und Typographie
% ===========================================================
	\usepackage{sourcecodepro}
	\usepackage[default]{sourcesanspro}
	\usepackage{nimbusmononarrow}

\usepackage[babel=true,final,tracking=smallcaps]{microtype}
\DisableLigatures{encoding = T1, family = tt* }
\usepackage{ellipsis}
% ===========================================================

% Farben
% ===========================================================
	\usepackage[usenames,x11names,final,table]{xcolor}
	\definecolor{fbblau}{HTML}{3078AB}
	\definecolor{mediumgray}{gray}{.65}
	\definecolor{blackberry}{rgb}{0.53, 0.0, 0.25}
% ===========================================================

% TikZ
% ===========================================================
	\usepackage{tikz}
	\usepackage{tikz-cd}					% kommutative Diagramme
	\usetikzlibrary{arrows.meta}			% mehr Pfeile!
	\usetikzlibrary{shadows}
	\usetikzlibrary{calc}
	\tikzset{>=Latex}						% Standard-Pfeilspitze
% ===========================================================

% Seitenlayout, Kopf-/Fußzeile
% ===========================================================
	\usepackage{scrpage2}
	\pagestyle{scrheadings}
	\usepackage[top=3cm, bottom=3cm, left=3cm, right=3cm]{geometry}
	\clearscrheadfoot 
	\setheadsepline{1pt} 					% Linie in Kopfzeile
	\automark[section]{section}				% Abschnittstitel in Kopfzeile
	\lehead{Camera Trap Challenge}
	\rohead{\rightmark}
	\lefoot{\pagemark}
	\rofoot{\pagemark}	% Seitenzahl in Fußzeile
	
	\raggedbottom
	\usepackage{setspace}					% Zeilenabstand 1.5-fach
	\setstretch{1.2}
	\setlength{\parindent}{0cm}
	\setlength{\parskip}{0.5\baselineskip}
	\usepackage[tikz]{mdframed}
	\usepackage[all]{nowidow}
% ===========================================================

% Hyperref
% ===========================================================
	\PassOptionsToPackage{hyphens}{url}
	\usepackage[%
		hidelinks,
		pdfpagelabels,
		bookmarksopen=true,
		bookmarksnumbered=true,
		linkcolor=black,
		urlcolor=SkyBlue2,
		plainpages=false,
		pagebackref,
		citecolor=black,
		hypertexnames=true,
		pdfauthor={Phil Steinhorst},
		pdfborderstyle={/S/U},
		linkbordercolor=SkyBlue2,
		colorlinks=false,
		backref=false]{hyperref}
	\renewcommand{\UrlFont}{\ttfamily}
	\urlstyle{tt}
	\hypersetup{final}
% ===========================================================

% Marginnotes / ToDo-Notes
% ===========================================================
	\usepackage[fulladjust]{marginnote}
	\renewcommand*{\marginfont}{\itshape\footnotesize}
	\usepackage[textsize=small,color=Red1!80!OrangeRed1!80]{todonotes}
% ===========================================================

% Listen und Tabellen
% ===========================================================
	\usepackage{multicol}
	\usepackage{multirow}
	\usepackage[shortlabels]{enumitem}
	\setlist{itemsep=0pt}
	\setlist[enumerate]{font=\sffamily\bfseries}
	\setlist[itemize]{label=$\triangleright$,font=\bfseries}
	\usepackage{tabularx}
% ===========================================================

% Inhaltsverzeichnis und Querverweise
% ===========================================================
	\usepackage[tocindentauto]{tocstyle}
	\usetocstyle{KOMAlike}	
	\usepackage[nameinlink]{cleveref}
	\usepackage{chngcntr}
	\usepackage[font=footnotesize]{caption}
% ===========================================================


% Sonstiges / für Testzwecke
% ===========================================================
	\usepackage{lipsum}
	\usepackage{pdfpages}
	\usepackage{eurosym}
% ===========================================================

\usepackage[natbib,style=alphabetic,minnames=2,maxnames=2,uniquelist=false,backend=biber]{biblatex}
\bibliography{bibliography}
\setlength{\bibitemsep}{1.5em}  

% minted
% ===========================================================
\usepackage{minted}
\setminted{%
	style=bw,
	fontsize=\normalsize,
	breaklines,
	breakanywhere=false,
	breakbytoken=false,
	breakbytokenanywhere=false,
	breakafter={.,},
	autogobble,
	numbersep=3mm,
	tabsize=4,
	frame=lines
}
\setmintedinline{%
	fontsize=\normalsize,
	numbers=none,
	numbersep=12pt,
	tabsize=4,
}
% ===========================================================


% mdframed
% ===========================================================
\usepackage[tikz]{mdframed}
\newmdenv[%
	backgroundcolor = gray!20,
	linewidth=0pt,
	skipabove = 0.8cm
	]{mybox}


% ===========================================================

% Abkürzungen
% ===========================================================
	\newcommand{\BB}{\mathbb{B}}
	\newcommand{\CC}{\mathbb{C}}
	\newcommand{\EE}{\mathbb{E}}
	\newcommand{\FF}{\mathbb{F}}
	\newcommand{\HH}{\mathcal{H}}
	\newcommand{\KK}{\mathbb{K}}
	\newcommand{\LL}{\mathbb{L}}
	\newcommand{\NN}{\mathbb{N}}
	\newcommand{\QQ}{\mathbb{Q}}
	\newcommand{\RR}{\mathbb{R}}
	\newcommand{\ZZ}{\mathbb{Z}}
	\newcommand{\oh}{\mathcal{O}}				% Landau-O
	\newcommand{\ind}{1\hspace{-0,8ex}1} 		% Indikatorfunktion (Doppeleins)
	\newcommand{\bewrueck}{\enquote{$\Leftarrow$}:} 	% Beweis Rückrichtung
	\newcommand{\bewhin}{\enquote{$\Rightarrow$}:}		% Beweis Hinrichtung
	\newcommand{\ol}[1]{\overline{#1}}
	\newcommand{\wt}[1]{\widetilde{#1}}
	\newcommand{\wh}[1]{\widehat{#1}}
% ===========================================================

% Operatoren
% ===========================================================
	\DeclareMathOperator{\id}{id} 				% Identität
	\DeclareMathOperator{\im}{im} 				% image
	\DeclareMathOperator{\pot}{\mathcal{P}}		% Potenzmenge
	\DeclareMathOperator{\sgn}{sgn} 			% Signum
	\DeclareMathOperator{\Sym}{Sym} 			% Symmetrische Gruppe
% ===========================================================

% Klammerungen und ähnliches
% ===========================================================
	\DeclarePairedDelimiter{\absolut}{\lvert}{\rvert}		% Betrag
	\DeclarePairedDelimiter{\ceiling}{\lceil}{\rceil}		% aufrunden
	\DeclarePairedDelimiter{\Floor}{\lfloor}{\rfloor}		% aufrunden
	\DeclarePairedDelimiter{\Norm}{\lVert}{\rVert}			% Norm
	\DeclarePairedDelimiter{\sprod}{\langle}{\rangle}		% spitze Klammern
	\DeclarePairedDelimiter{\enbrace}{(}{)}					% runde Klammern
	\DeclarePairedDelimiter{\benbrace}{\lbrack}{\rbrack}	% eckige Klammern
	\DeclarePairedDelimiter{\penbrace}{\{}{\}}				% geschweifte Klammern
	\newcommand{\Underbrace}[2]{{\underbrace{#1}_{#2}}} 	% bessere Unterklammerungen
	% Kurzschreibweisen für Faule und Code-Vervollständigung
	\newcommand{\abs}[1]{\absolut*{#1}}
	\newcommand{\ceil}[1]{\ceiling*{#1}}
	\newcommand{\flo}[1]{\Floor*{#1}}
	\newcommand{\no}[1]{\Norm*{#1}}
	\newcommand{\sk}[1]{\sprod*{#1}}
	\newcommand{\enb}[1]{\enbrace*{#1}}
	\newcommand{\penb}[1]{\penbrace*{#1}}
	\newcommand{\benb}[1]{\benbrace*{#1}}
	\newcommand{\stack}[2]{\makebox[1cm][c]{$\stackrel{#1}{#2}$}}
% ===========================================================

% Monotypes
% ===========================================================
	\newcommand{\Band}{\mathtt{AND}}
	\newcommand{\Bor}{\mathtt{OR}}
	\newcommand{\zero}{\mathtt{0}}
	\newcommand{\one}{\mathtt{1}}
	\newcommand{\Bnot}{\mathtt{NOT}}
	\newcommand{\Bnand}{\mathtt{NAND}}
	\newcommand{\Bnor}{\mathtt{NOR}}
	\newcommand{\Bxor}{\mathtt{XOR}}
	\DeclareMathOperator{\DNF}{DNF}
	\DeclareMathOperator{\KNF}{KNF}