%!TEX root = ../vortrag.tex
\section{Fazit}
\begin{frame}[t,fragile]{Fazit -- Substitutionsmodell}
	\begin{mybox}
		Um eine Prozedur auf konkrete Werte anzuwenden, ersetze im Rumpf der Prozedur jeden formalen Parameter mit dem entsprechenden Wert und werte den Rumpf aus.
	\end{mybox}
	
	\vspace*{0.5cm}
	
	Das Substitutionsmodell\dots
	\begin{itemize}
		\item macht den Auswertungsprozess von Prozeduren inklusive der Weitergabe von Rückgabewerten erschließbar,
		\item ist dabei vergleichsweise intuitiv und leicht nachvollziehbar,
		\item besitzt einen vertretbaren Aufwand bei der Anwendung -- auch bei verschachtelten Prozeduraufrufen wie z.B. Rekursionen,
		\item hat Schwierigkeiten mit der Verwendung imperativer Elemente (Zustandsabhängigkeit),
		\item berücksichtigt nicht das Konzept lokaler Variablen.
	\end{itemize}
\end{frame}

\begin{frame}[t,fragile]{Fazit -- Umgebungsmodell}
	\begin{mybox}
		Um eine Prozedur auszuwerten, die in der Umgebung $\mathcal{U}$ erzeugt wurde, verfahre wie folgt:
		\begin{enumerate}
			\item Erzeuge eine neue Umgebung mit neuem Bindungsrahmen, der an die Umgebung $\mathcal{U}$ angehängt wird.
			\item Binde die formalen Parameter im erstellten Rahmen an die Aufrufparameter.
			\item Werte den Rumpf in der erzeugten Umgebung aus.
		\end{enumerate}
	\end{mybox}
	
	\vspace*{0.5cm}
	
	Das Umgebungsmodell\dots
	\begin{itemize}
		\item erklärt das Konzept lokaler Variablen und deren Sichtbarkeit,
		\item ist kompatibel mit funktionalen und imperativen Elementen,
		\item kann bei verschachtelten Funktionsaufrufen schnell unübersichtlich werden,
		\item lässt Rückgabewerte unberücksichtigt.
	\end{itemize}
\end{frame}

\begin{frame}{}
	\begin{center} \small
		\includegraphics[keepaspectratio,width=13cm]{img/xkcd-297.png}
		
		\url{https://xkcd.com/297/}
	\end{center}
\end{frame}

\nocite{*}
\section*{Quellen}
\begin{frame}[allowframebreaks,t]{\secname}
	\printbibliography
\end{frame}